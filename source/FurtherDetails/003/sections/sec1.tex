\section{Introduction}
\label{sec:intro}

This document specifies the software standards and coding styles to be used when writing new 
code files for the Met Office Unified Model. When making {\bf extensive} changes to an 
existing file a {\bf rewrite} of the whole file should be done to ensure that the file meets 
the UM coding standard and style. {\bf All code modifications within an existing file 
should follow these standards}.

The only exception to following these coding standards is that there is no requirement 
to rewrite 'imported code' to these standards before it is included within the UM. 
All new code developed within the Met Office should follow these standards.

Imported code; is developed as part of a collaboration project 
and then proposed to be suitable for use within the UM; for example the original 
UKCA code developed in academia. Collaborative developed code specifically for the 
UM should meet these standards.

\subsection{Why have standards?}

This document is intended for new as well as 
experienced programmers, so a few words about why there is a need for 
software standards and styles may be in order. 

Coding standards specify a standard working practice for a project 
with the aim of improving portability, maintainability and the readability of code. 
This process makes code development and reviewing easier for all developers 
involved in the project. Remember that software should be written for people 
and not just for computers! As long as the syntax rules of the programming language 
(e.g. Fortran IV -- 2003) are followed, the computer does not care how the code is written. 
You could use archaic language structures, add no comments, leave no spaces etc. 
However, another programmer trying to use, maintain or alter the code will have 
trouble working out what the code does and how it does it. 
A little extra effort whilst writing the code can greatly simplify the task of 
this other programmer (which might be the original author a year or so after 
writing the code, when details of it are bound to have been forgotten). 
In addition, following these standards may well help you to write better, 
more efficient, programs containing fewer bugs.

While code style is very subjective, by standardising the style, UM routine
layout will become familiar to all code developers/reviewers even when 
they are not familiar with the underlying science.

\subsection{Units}

All routines and documentation must be written using SI units. 
Standard SI prefixes may be used. Where relevant, 
the units used must be clearly stated in both the code and the supporting UM documentation.  

\subsection{Working practices}

The preparation of new files and of changes to existing files should, 
meet this UM standard documentation and must be developed following the stages outlined in `` 
\href{https://code.metoffice.gov.uk/trac/um/wiki/working_practices}{Working Practices for UM Development under FCM}''.

\subsection{Examples}

This document provides an example programming unit to aid the code developer.
This example meets the standards detailed within this paper, with
references to the relevant sections.


\subsection{Technical standards}
UM code should be written in and conform to the Fortran 2003 standard;
this is supported by most
\href{http://fortranwiki.org/fortran/show/Fortran+2003+status}
{major Fortran compilers}. Obsolescent language features are not permitted.
The UM also requires compiler support for
Technical Specification 29113 on the Further Interoperability of Fortran with C.
This is a new feature for Fortran 2018, but is a
\href{http://fortranwiki.org/fortran/show/Compiler+Support+for+Modern+Fortran}
{common extension in most compilers} and has widespread support.

Please note that in order to maximise portability and to avoid the use of
radically different design structures within single areas of code, some Fortran
2003 features are excluded from use within the UM. For further details please
see Appendix \ref{app:F2003}.


\subsection{Pre-processor}
In the past include files and C pre-processor were used for scientific code section choices and passing a large list of arrays. 
This use has been phased out and highly discouraged. 
The C pre-processor is still used to make machine specific choices and, together with included files, to  
reduced code duplication. These are all covered by this standards and style document. 

